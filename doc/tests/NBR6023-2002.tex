%% Copyright 2016 Daniel Ballester Marques
%%
%% This work may be distributed and/or modified under the
%% conditions of the LaTeX Project Public License, either version 1.3
%% of this license or (at your option) any later version.
%% The latest version of this license is in
%%   http://www.latex-project.org/lppl.txt
%% and version 1.3 or later is part of all distributions of LaTeX
%% version 2005/12/01 or later.
%%
%% This work has the LPPL maintenance status `maintained'.
%%
%% The Current Maintainer of this work is Daniel Ballester Marques.

\documentclass[a4paper]{article}
\usepackage[brazil]{babel}
\usepackage{lmodern}
\usepackage[T1]{fontenc}
\usepackage[utf8]{inputenc}
\usepackage{textcomp}
\usepackage{microtype}
\usepackage{hyperref}

\usepackage[
    backend=biber,
    style=abnt,
    sorting=none,
    repeatfirstfields,
    indent,
]{biblatex}
\usepackage[autostyle]{csquotes}

\addbibresource{NBR6023-2002.bib}

\defbibheading{bibliography}[]{
    \subsection{#1}
}
\defbibenvironment{bibliography}
  {\list{}{%
     \leftmargin 6em
     \itemsep\bibitemsep
     \parsep\bibparsep}}
  {\endlist}
  {\item}

\defbibheading{subbib}[]{
    \subsubsection{#1}
}

\title{NBR 6023:2002 - Testes}
\author{Daniel Ballester Marques}


\begin{document}

\maketitle

Este documento imprime \textit{ipsis litteris} todas as entradas citadas na
NBR 6023:2002, da ABNT.

Para expor as diferentes maneiras como cada campo pode ser usado, a ABNT
imprime-os de modo diferente a cada entrada. A vantagem dos sistemas como
o \texttt{biblatex}, por outro lado, é \emph{uniformizar} a bibliografia,
imprimindo os campos do mesmo modo em todas as entradas. Para reproduzir
fielmente o texto do manual, este documento faz \emph{mau uso} dos recursos do
\texttt{biblatex}, aproveitando-se frequentemente de campos que devem ser
evitados, como \texttt{note}, \texttt{addendum}, \texttt{nameaddon},
\texttt{titleaddon} et al. Ele não deve ser usado, portanto, como referência
para o uso do \texttt{biblatex-abnt}, mas apenas como exemplo de suas
capacidades, mostrando que é possível (embora não seja recomendado) reproduzir
todas as entradas do manual e expondo o modo de fazê-lo. Para o uso correto do
\texttt{biblatex-abnt}, conferir o manual do estilo e o manual do próprio
\texttt{biblatex}.

\tableofcontents

\nocite{*}

\clearpage


\setcounter{section}{6}


\section{Modelos de referências}


	\printbibliography[keyword=7.1, title={Monografia no todo}]

	\printbibliography[keyword=7.2,
    title={Monografia no todo em meio eletrônico}]

	\printbibliography[keyword=7.3, title={Parte de monografia}]

	\printbibliography[keyword=7.4,
    title={Parte de monografia em meio eletrônico}]

	\subsection{Publicação periódica}

	  \printbibliography[heading=subbib, keyword=7.5.1,
      title={Publicação periódica como um todo}]
	  \printbibliography[heading=subbib, keyword=7.5.2,
      title={Partes de revista, boletim etc.}]
	  \printbibliography[heading=subbib, keyword=7.5.3,
      title={Artigo e/ou matéria de revista, boletim etc.}]
	  \printbibliography[heading=subbib, keyword=7.5.4,
      title={Artigo e/ou matéria de revista,
             boletim etc.\ em meio eletrônico}]
	  \printbibliography[heading=subbib, keyword=7.5.5,
      title={Artigo e/ou matéria de jornal}]

	   \printbibliography[heading=subbib, keyword=7.5.6,
      title={Artigo e/ou matéria de jornal em meio eletrônico}]

	\subsection{Evento como um todo}

    \printbibliography[heading=subbib, keyword=7.6.1,
      title={Elementos essenciais}]
    \printbibliography[heading=subbib, keyword=7.6.2,
      title={Elementos complementares}]
    \printbibliography[heading=subbib, keyword=7.6.3,
      title={Evento como um todo em meio eletrônico}]

	\subsection{Trabalho apresentado em evento}

    \printbibliography[heading=subbib, keyword=7.7.1,
      title={Elementos essenciais}]
    \printbibliography[heading=subbib, keyword=7.7.2,
      title={Elementos complementares}]
    \printbibliography[heading=subbib, keyword=7.7.3,
      title={Trabalho apresentado em evento em meio eletrônico}]

	\printbibliography[keyword=7.8, title={Patente}]

	\subsection{Documento jurídico}

    \printbibliography[heading=subbib, keyword=7.9.1, title={Legislação}]
    \printbibliography[heading=subbib, keyword=7.9.2,
      title={Jurisprudência (decisões judiciais)}]
    \printbibliography[heading=subbib, keyword=7.9.3, title={Doutrina}]
    \printbibliography[heading=subbib, keyword=7.9.4,
      title={Documento jurídico em meio eletrônico}]

	\printbibliography[keyword=7.10, title={Imagem em movimento}]

	\subsection{Documento iconográfico}

    \printbibliography[heading=subbib, keyword=7.11.1,
      title={Elementos essenciais}]
    \printbibliography[heading=subbib, keyword=7.11.2,
      title={Elementos complementares}]
    \printbibliography[heading=subbib, keyword=7.11.3,
      title={Documento iconográfico em meio eletrônico}]

	\subsection{Documento cartográfico}

    \printbibliography[heading=subbib, keyword=7.12.1,
      title={Elementos essenciais}]
    \printbibliography[heading=subbib, keyword=7.12.2,
      title={Elementos complementares}]
    \printbibliography[heading=subbib, keyword=7.12.3,
      title={Documento cartográfico em meio eletrônico}]

	\printbibliography[keyword=7.13, title={Documento sonoro no todo}]

	\printbibliography[keyword=7.14, title={Documento sonoro em parte}]

	\subsection{Partitura}

    \printbibliography[heading=subbib, keyword=7.15.1,
      title={Elementos essenciais}]
    \printbibliography[heading=subbib, keyword=7.15.2,
      title={Elementos complementares}]
    \printbibliography[heading=subbib, keyword=7.15.3,
      title={Partitura em meio eletrônico}]

	\printbibliography[keyword=7.16, title={Documento tridimensional}]

	\printbibliography[keyword=7.17,
    title={Documento de acesso exclusivo em meio eletrônico}]

	\section{Transcrição dos elementos}

	\subsection{Autoria}

    \printbibliography[heading=subbib, keyword=8.1.1, title={Autor pessoal}]
    \printbibliography[heading=subbib, keyword=8.1.2, title={Autor entidade}]
    \printbibliography[heading=subbib, keyword=8.1.3,
      title={Autoria desconhecida}]

	\printbibliography[keyword=8.2, title={Título e subtítulo}]

	\printbibliography[keyword=8.3, title={Edição}]

	\printbibliography[keyword=8.4, title={Local}]

	\printbibliography[keyword=8.5, title={Editora}]

	\printbibliography[keyword=8.6, title={Data}]

	\printbibliography[keyword=8.7, title={Descrição física}]

	\printbibliography[keyword=8.8, title={Ilustrações}]

	\printbibliography[keyword=8.9, title={Dimensões}]

	\printbibliography[keyword=8.10, title={Séries e coleções}]

	\printbibliography[keyword=8.11, title={Notas}]

	\section{Ordenação das referências}

	\printbibliography[keyword=9.1, title={Sistema alfabético}]

	\printbibliography[keyword=9.2, title={Sistema numérico}]


\end{document}
