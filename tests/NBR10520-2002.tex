%% Copyright 2016 Daniel Ballester Marques
%%
%% This work may be distributed and/or modified under the
%% conditions of the LaTeX Project Public License, either version 1.3
%% of this license or (at your option) any later version.
%% The latest version of this license is in
%%   http://www.latex-project.org/lppl.txt
%% and version 1.3 or later is part of all distributions of LaTeX
%% version 2005/12/01 or later.
%%
%% This work has the LPPL maintenance status `maintained'.
%%
%% The Current Maintainer of this work is Daniel Ballester Marques.

\documentclass[a4paper]{article}
\usepackage[brazil]{babel}
\usepackage{lmodern}
\usepackage[T1]{fontenc}
\usepackage[utf8]{inputenc}
\usepackage{textcomp}
\usepackage{microtype}
\usepackage{etoolbox}
\usepackage{xcolor}

\usepackage[backend=biber, style=abnt]{biblatex}
\usepackage[autostyle]{csquotes}

\addbibresource{NBR10520-2002.bib}

\setlength{\parskip}{\baselineskip}%
\setlength{\parindent}{0pt}%
\emergencystretch=1em

\makeatletter
\newcommand{\globalcolor}[1]{%
  \color{#1}\global\let\default@color\current@color
}
\makeatother
\AtBeginDocument{\iftoggle{reference}{}{\globalcolor{red!50}}}

\usepackage{hyperref}


\newtoggle{reference}
\toggletrue{reference}

\title{NBR 10520:2002 - \iftoggle{reference}{Referência}{Teste}}
\author{Daniel Ballester Marques}

\begin{document}

\frenchspacing

\maketitle

Este documento imprime \textit{ipsis litteris} todas as entradas citadas na
NBR 10520:2002 usando o sistema autor-data.

\tableofcontents

\clearpage

\setcounter{section}{5}

\section{Regras gerais de apresentação}

A ironia seria assim uma forma implícita de heterogeneidade mostrada, conforme
a classificação proposta por \textcite{authier1982}.

\enquote{Apesar das aparências, a desconstrução do logocentrismo não é uma
psicanálise da filosofia [\ldots]} \cite[293]{derrida1967}.

A produção de lítio começa em Searles Lake, Califórnia, em 1928
\cite[513]{mumford1949}.

\textcite[146]{oliveira1943} dizem que a \enquote{[\ldots] relação da série
São Roque com os granitos porfiróides pequenos é muito clara.}

Meyer parte de uma passagem da crônica de \enquote{14 de maio}, de A Semana:
\enquote{Houve sol, e grande sol, naquele domingo de 1888, em que o Senado
votou a lei, que a regente sancionou [\ldots]} \cite[v. 3, p. 583]{assis1994}.

\textcite[35]{barbour1971} descreve: \enquote{O estudo da morfologia dos
terrenos [\ldots] ativos [\ldots]}

\enquote{Não se mova, faça de conta que está morta.} \cite[72]{clarac1985}.

Segundo \textcite[27]{sa1995}: \enquote{[\ldots] por meio da mesma \enquote{arte de
conversação} que abrange tão extensa e significativa parte da nossa existência
cotidiana [\ldots]}

A teleconferência permite ao indivíduo participar de um encontro nacional ou
regional sem a necessidade de deixar seu local de origem. Tipos comuns de
teleconferência incluem o uso da televisão, telefone, e computador. Através de
áudio-conferência, utilizando a companhia local de telefone, um sinal de áudio
pode ser emitido em um salão de qualquer dimensão. \cite[181]{nichols1993}.

\enquote{[\ldots] para que não tenha lugar a producção de degenerados, quer
physicos quer moraes, misérias, verdadeiras ameaças à sociedade.} \cite[p. 46,
grifo nosso]{souto1916}.

\enquote{[\ldots] b) desejo de criar uma literatura independente, diversa, de
vez que, aparecendo o classicismo como manifestação de passado colonial
[\ldots]} \cite[v. 2, p. 12, grifo do autor]{candido1993}.

\enquote{Ao fazê-lo pode estar envolto em culpa, perversão, ódio de si mesmo
[\ldots] pode julgar-se pecador e identificar-se com seu pecado.} \cite[v. 4,
p. 463, tradução nossa]{rahner1962}.


\section{Sistema de chamada}

Em \citetitle{teatro1963} relata-se a emergência do teatro do absurdo.

Segundo \textcite[32]{morais1955} assinala \enquote{[\ldots] a presença de
concreções de bauxita no Rio Cricon.}

\cite{barbosa1958}

\cite{barbosa1959}

\cite{barbosa1965a}

\cite{barbosa1965b}

De acordo com \textcite{reeside1927a}

\cite{reeside1927b}

\cites{dreyfuss1989}{dreyfuss1991}{dreyfuss1995}

\cites{cruz1998}{cruz1999}{cruz2000}

Ela polariza e encaminha, sob a forma de \enquote{demanda coletiva}, as
necessidades de todos \cites{fonseca1997}{paiva1997}{silva1997}.

Diversos autores salientam a importância do \enquote{acontecimento
desencadeador} no início de um processo de aprendizagem
\cites{cross1984}{knox1986}{mezirow1991}.

A chamada \enquote{pandectística havia sido a forma particular pela qual
o direito romano fora integrado no século XIX na Alemanha em particular.}
\cite[225]{lopes2000}.

\textcite[30]{bobbio1995} com muita propriedade nos lembra, ao comentar esta
situação, que os \enquote{juristas medievais justificaram formalmente
  a validade do direito romano ponderando que este era o direito do Império
  Romano que tinha sido reconstituído por Carlos Magno com o nome de Sacro
Império Romano.}

De fato, semelhante equacionamento do problema conteria o risco de se
considerar a literatura meramente como uma fonte a mais de conteúdos já
previamente disponíveis, em outros lugares, para a teologia
\cite[3]{jossua1976}.

\textcite{merriam1991} observam que a localização de recursos tem um papel
crucial no processo de aprendizagem autodirigida.

\enquote{Comunidade tem que poder ser intercambiada em qualquer circunstância,
sem quaisquer restrições estatais, pelas moedas dos outros Estados-membros.}
\cite[34]{comissao1992}.

O mecanismo proposto para viabilizar esta concepção é o chamado Contrato de
Gestão, que conduziria à captação de recursos privados como forma de reduzir
os investimentos públicos no ensino superior \cite{brasil1995}.

\enquote{As IES implementarão mecanismos democráticos, legítimos
e transparentes de avaliação sistemática das suas atividades, levando em conta
seus objetivos institucionais e seus compromissos para com a sociedade.}
\cite[55]{anteprojeto1987}.

E eles disseram \enquote{globalização}, e soubemos que era assim que chamavam
a ordem absurda em que dinheiro é a única pátria à qual se serve e as
fronteiras se diluem, não pela fraternidade, mas pelo sangramento que engorda
poderosos sem nacionalidade. \cite[4]{aflor1995}.

\enquote{Em Nova Londrina (PR), as crianças são levadas às lavouras a partir
dos 5 anos.} \cite[12]{noscanaviais1995}.


\section{Notas de rodapé}

Segundo \textapud{silva1983}[3]{abreu1999} diz ser [\ldots]

\enquote{[\ldots] o viés organicista da burocracia estatal e o antiliberalismo
da cultura política de 1937, preservado de modo encapuçado na Carta de 1946.}
\apud[172]{vianna1986}[214-215]{segatto1995}.

No modelo serial de \textapud{gough1972}{nardi1993}, o ato de ler envolve um
processamento serial que começa com uma fixação ocular sobre o texto,
prosseguindo da esquerda para a direita de forma linear.

\setcounter{footnote}{3}
Os pais estão sempre confrontados diante das duas alternativas: vinculação
escolar ou vinculação profissional.\footcite[Sobre essa opção dramática, ver
também][269--290]{morice1996}

\end{document}

